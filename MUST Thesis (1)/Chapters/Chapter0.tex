\chapter{Удиртгал}
\newpage
\section{Сэдэв сонгосон үндэслэл, судалгааны асуудлыг тодорхойлох}
Хэрэглэгчдэд хялбар, шуурхай худалдаа хийх боломжийг олгох, цагийн хуваарьт суурилсан автомат худалдааг цахим хэлбэрээр зохион байгуулах нь уламжлалт, нэгэн хэвийн худалдааны системүүдээс илүү үр дүнтэй, хүртээмжтэй, уян хатан шийдэл болох нь ажиглагдаж байна. Олон нийтийн хэрэгцээ, техник технологийн хөгжил, хэрэглэгчдийн хандлага өөрчлөгдөж буй энэ үед цахим дуудлага худалдааны систем нь орчин үеийн шаардлагад нийцсэн, байгальд ээлтэй, хаягдал багатай, илүү зохион байгуулалттай худалдааны арга хэлбэр юм. Судалгааны үндсэн асуудал нь хэрхэн хэрэглэгчдэд ээлтэй, найдвартай, автомат ажиллагаатай, дахин ашиглалт, хэрэглээнд суурилсан цахим дуудлага худалдааны системийг хөгжүүлэхэд оршино.

\section{Зорилго}
Цахим дуудлага худалдааны системийг хөгжүүлснээр худалдагч болон худалдан авагч нарт хэрэглэхэд хялбар, найдвартай, хүртээмжтэй орчинг бүрдүүлэх боломжтой болно. Энэхүү систем нь худалдан авагчид болон борлуулагчдыг шууд холбож, аюулгүй, ил тод дуудлага худалдааны үйл ажиллагааг дэмжих зорилготой бөгөөд ингэснээр худалдааны процессыг илүү үр дүнтэй, найдвартай болгоно. Мөн цахим дуудлага худалдааны систем нь хуучин бараа, эд зүйлсийн дахин ашиглалт, шилжүүлэн борлуулалтыг дэмжснээр хог хаягдлын хэмжээ буурахад өндөр хувь нэмэр оруулдаг. Ийм байдлаар хуучин эд зүйлс дахин хэрэглэгдэх боломжтой болж, хог хаягдлын ачааллыг багасгах төдийгүй байгаль орчны бохирдлыг бууруулах, нөөц баялгийг үр ашигтай ашиглахад чухал үүрэг гүйцэтгэнэ. Тиймээс цахим дуудлага худалдааны систем нь орчин үеийн нийгэмд эдийн засаг, байгаль орчны тогтвортой хөгжилд дэмжлэг үзүүлэх чухал хэрэгсэл болж байгаа юм.

\section{Зорилт}
\begin{itemize}
    \item Real-time data ашиглаж мэдээлэл солилцох боломжийг бүрдүүлэх.
    \item Төлбөрийн системийг Qpay ашиглаж байгуулах.
    \item Хэрэглэгчийн нэвтрэх бүртгүүлэх хэсгийг gmail-тэй холбох.
    \item Дуудлага худалдааны үйл ажиллагааг автоматжуулах, бодит цагийн мэдээлэл дамжуулах.
    \item MongoDB ашиглаж өгөгдлүүдийг найдвартай хадгалах.
    \item Системийн аюулгүй байдал, гүйцэтгэлийг сайжруулах.
    \item Туршилт, тест хийх замаар системийн найдвартай ажиллагааг хангах.
\end{itemize}

\section{Судалгааны хамрах хүрээ}
Энэхүү систем нь Монгол орны аль ч бүс нутагт оршин суугаа хүн бүрд тохиромжтой. Ялангуяа өөрт хэрэггүй болсон бараа, эд зүйлсийг бусдад худалдаалах сонирхолтой хувь хүн, жижиг дунд бизнес эрхлэгчид, мөн хямд үнээр шаардлагатай барааг худалдан авах хүсэлтэй хэрэглэгчид гол хэрэглэгчид болно. Систем нь олон нийтийн түвшинд өргөн хүрээнд ашиглагдах боломжтой бөгөөд цахим худалдааны хүртээмжийг нэмэгдүүлнэ.

\section{Ач холбогдол}
Онлайн дуудлага худалдааны систем нь хувь хүн болон жижиг дунд бизнес эрхлэгчдэд дараах хэд хэдэн ач холбогдолтой:

\begin{itemize}
\item Түрээсийн зардал өндөртэй уламжлалт худалдааны хэлбэрээс татгалзаж, зардлыг бууруулах боломжийг олгоно.
\item Өөрт хэрэггүй болсон бараа, эд зүйлсээ бусдад цахим орчинд хялбар аргаар борлуулах боломжтой болсноор эдийн засгийн эргэлтэд оруулах нөхцөл бүрдэнэ.
\item Хуучин эд зүйлсийг дахин ашиглуулах, дахин хэрэглэх боломжийг бүрдүүлснээр хог хаягдлыг бууруулахад өндөр хувь нэмэр оруулна. Энэ нь байгаль орчны ачааллыг бууруулж, тогтвортой хөгжилд чиглэсэн бодит алхам юм.
\item Хэрэглэгчдийн цаг хугацаа, зардлыг хэмнэж, илүү хурдан шуурхай аргаар шаардлагатай бараагаа олох боломжийг нэмэгдүүлнэ.
\item Бизнес эрхлэгчдэд илүү өргөн хүрээний зах зээлд хүрэх, хэрэглэгчдийн мэдээлэлд үндэслэн борлуулалтаа оновчтой болгох боломжийг олгоно.
\end{itemize}
\section{Гарах үр дүн}
\begin{itemize}
    \item Дуудлага худалдааны үйл явцыг бүрэн автоматжуулсан цахим систем бий болно.
    \item Худалдан авагч, борлуулагчдын хоорондын харилцааг хялбаршуулсан орчин бүрдэнэ.
    \item Бодит цагийн мэдээлэл дамжуулдаг, ил тод, найдвартай систем хөгжүүлэгдэнэ.

    \item Ашиглахад хялбар хэрэглэгчийн интерфейстэй платформ бий болно.
    \item Системийн аюулгүй байдал, өгөгдлийн хамгаалалт сайжирна.
    \item Дуудлага худалдааны зах зээлийн үйл ажиллагааг дэмжих технологийн шийдэл бий болно.
\end{itemize}